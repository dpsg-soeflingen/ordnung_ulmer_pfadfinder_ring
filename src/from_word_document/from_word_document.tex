LEITBILD
\\

Das große Ziel der pfadfinderischen Arbeit ist es, Kinder und Jugendliche zu stärken und sie zu 
befähigen, ihre Potenziale so auszuschöpfen, dass sie als verantwortungsbewusste Bürgerinnen und 
Bürger die Welt mitgestalten können. Folgende Ziele verfolgt der Ulmer Pfadfinder-Ring:
\\

"look at the boy" – "look at the girl" – Das Zitat "look at the child" von Lord Robert Baden-Powell, 
dem Gründer der Weltpfadfinderbewegung, fasst die Grundlage für die Arbeit mit Kindern und 
Jugendlichen nach dem pfadfinderischen Verständnis zusammen: Der Entwicklungsstand und die 
Lebenswirklichkeit der Kinder und Jugendlichen werden innerhalb der Gruppenarbeit stets 
berücksichtigt. Im Grundsatz "look at the girl" betrachten die Mitglieder im Ring Deutscher 
Pfadfinderinnenverbände speziell die Mädchen und jungen Frauen.
\\

"Learning by doing" – In unseren Gruppen entwickeln Kinder und Jugendliche ihre Fähigkeiten durch 
eigenes Ausprobieren, durch Mitbestimmung und eigene Entscheidungen. Sie lernen durch die Übernahme 
von Aufgaben Verantwortung zu tragen, sie lernen partnerschaftlich zu handeln und sich über 
gemeinsame Erfolge zu freuen. Gleichzeitig erlernen sie soziales Verhalten durch das Akzeptieren 
von (Spiel-)Regeln.
\\

Demokratie lernen im Handeln – In vielen Projekten befähigt die pfadfinderische Methode Kinder und 
Jugendliche, soziale und politische Zusammenhänge zu erkennen, sich zu orientieren und ihre 
Interessen solidarisch mit anderen zu vertreten – auf lokaler, nationaler und internationaler Ebene. 
In den unterschiedlichen Gremien der Ortsgruppen der Verbände – der sogenannten Stämme – und ihrer 
Dachorganisationen wird Interessenvertretung und politisches Handeln nicht nur von Leiterinnen und 
Leitern, sondern schon von den Jüngsten eingeübt.
\\

Gerechtigkeit leben – Pfadfinder*innen treten für eine gerechte Welt ein. Dabei fangen sie zuerst 
bei sich an, vergessen dabei aber nicht die Menschen in anderen Ländern. Mädchen und Jungen sind bei 
den Pfadfinder*innen selbstverständlich gleichberechtigt und treten darüber hinaus dafür ein, dass 
sie in der Gesellschaft gleiche Chancen haben.
\\

Natur und Umwelt – Ein Höhepunkt im pfadfinderischen Jahreslauf findet draußen unter freiem Himmel 
statt: In Zeltlagern und "auf Fahrt" leben sie in und mit der Natur. Hier erleben Kinder und 
Jugendliche, mit wie wenig sie auskommen können und was wirklich von Bedeutung ist. So sind 
natürlich auch Nachhaltigkeit und Umweltschutz zwei große Themen, die bereits den Kleinen im 
Gruppenalltag beispielsweise bei einer Müllsammelaktion durch aktives Handeln nahegebracht werden.
\\

Pfadfinden für alle – Der Ulmer Pfadfinder-Ring steht für gegenseitige Akzeptanz, für die Erziehung 
zum Frieden und den weltweiten Abbau von Ungerechtigkeit und Armut. Darum sind bei den Gruppen des 
Ulmer Pfadfinder-Ring Kinder und Jugendliche jeder nationalen, religiösen, ethnischen oder sozialen 
Zugehörigkeit herzlich willkommen.

\newpage

Satzung
\\

1. Name und Geschäftsjahr der Interessengemeinschaft

1.1 Der Name der Interessengemeinschaft ist "Ulmer Pfadfinder-Ring".

1.2 Das Geschäftsjahr der Interessengemeinschaft entspricht dem Kalenderjahr.

2. Zweck der Interessengemeinschaft

2.1 Die Interessengemeinschaft ist ein Zusammenschluss von Pfadfinder-Gruppierungen zur Erziehung junger Menschen nach den Grundsätzen der Internationalen Pfadfinderbewegung im Zusammenhang mit allen Trägern der Jugenderziehung zu freien, verantwortungsbewussten und toleranten Bürgerinnen und Bürgern eines Demokratischen Staates.

2.2 Die Interessengemeinschaft ist parteipolitisch und konfessionell neutral. Sie bekennt sich zum demokratischen Rechtsstaat und zum Grundgesetz der Bundesrepublik Deutschland.

2.3 Veranstaltungen zum regionalen (auch nationalen und internationalen) Treffen von Kindern und Jugendlichen.

3. Organe der Interessengemeinschaft

3.1 Die Organe der Interessengemeinschaft sind

3.1.1 Der Sprecher oder die Sprecherin.

3.1.2 Die Delegiertenversammlung.

4. Sprecher oder Sprecherin

4.1 Der Sprecher oder die Sprecherin wird durch die Delegiertenversammlung demokratisch gewählt.

4.2 Die Wiederwahl des Sprechers oder der Sprecherin ist zulässig.

4.3 Bei Ausscheiden des Sprechers oder der Sprecherin wählt die Delegiertenversammlung eine Nachfolge.

4.4 Die Abwahl des Sprechers oder der Sprecherin ist auf Antrag durch die Delegiertenversammlung möglich.

4.4.1 Der Antrag gilt bei mindestens 75\% Zustimmung als Angenommen.

4.5 Der Sprecher oder die Sprecherin ist an die Beschlüsse der Delegiertenversammlung unter Beachtung der Satzung gebunden.

4.6 Der Sprecher oder die Sprecherin hat folgende Aufgaben:

4.6.1 Weitergabe von Terminen und Aufgaben an die Gruppen.

4.6.2 Terminfindung für Delegiertenversammlungen.

4.7 Der Sprecher oder die Sprecherin fungiert als Vermittler/-in zwischen dem Ulmer Pfadfinder-Ring und dem Stadtjugendring (SJR).

4.8 Der Sprecher oder die Sprecherin ist Ansprechpartner/-in und Vermittler/-in gegenüber dritten.

5. Delegiertenversammlung

5.1 Die Delegiertenversammlung ist das oberste beschlussfassende Organ.

5.2 Die Delegiertenversammlung entscheidet grundsätzlich mit einfacher Mehrheit der in der Versammlung abgegebenen Stimmen(Stimmenmehrheit).

5.3 Im Falle einer Stimmengleichheit gilt der Antrag als abgelehnt.

5.4 Die Delegiertenversammlung tritt mindestens einmal jährlich zusammen.

5.5 Die Delegiertenversammlung setzt sich aus jeweils zwei Vertretern oder Vertreterinnen aus den Mitgliedsgruppen zusammen.

5.5.1 Die Delegierten können von jeder Gruppe frei bestimmt werden.

5.5.2 Geschieht das nicht, sind die gewählten Vorstände der Gruppen Kraft Amtes, hierarchisch absteigend, delegiert.

5.6 Ein Mitglied der Delegiertenversammlung muss das 16. Lebensjahr vollendet haben.

5.7 Anträge

5.7.1 Jede Gruppe kann jederzeit in schriftlicher oder mündlicher Form Anträge anbringen.

5.7.2 Einfache Anträge werden in der folgenden Delegiertenversammlung bearbeitet.

5.7.3 Bei einem Eilantrag kann eine außerordentliche Delegiertenversammlung einberufen werden.

5.7.3.1 Ein Antrag ist eilig, wenn dieser einen unaufschiebbaren und den Pfadfinder-Ring nicht unerheblich betreffenden Inhalt vorweist.

6. Eintritt, Austritt und Ausschluss aus dem Ulmer Pfadfinder-Ring

6.1 Eintritt

6.1.1 Ein Aufnahmeantrag kann nur schriftlich erfolgen.

6.1.1.1 Der Antrag kann formlos gestellt werden.

6.1.2 Über Aufnahmeanträge entscheidet die Delegiertenversammlung.

6.1.2.1 Der Aufnahme-Antrag gilt bei mindestens 75\% Zustimmung als Angenommen.

6.1.3 Die antragstellende Gruppe muss sich vor Beitritt zu der Satzung des "Ulmer Pfadfinder-Ring" bekennen.

6.2 Austritt

6.2.1 Eine Austrittserklärung kann nur schriftlich erfolgen.

6.2.2 Bei Zustimmung durch die Delegiertenversammlung (Eilantrag) kann der Austritt sofort umgesetzt werden.

6.2.3 Andernfalls gilt die nächste ordentliche Delegiertenversammlung als Austrittstermin.

6.3 Ausschluss

6.3.1 Ein Ausschluss kann erfolgen, wenn:

6.3.1.1 Die Gruppe nicht mehr existiert

6.3.1.2 Die Gruppe seit mindestens einem Jahr keine Angebote für Kinder und Jugendliche anbietet.

6.3.1.3 Die Arbeit der Gruppe dem Leitbild oder der Satzung des Ulmer Pfadfinder-Rings widerspricht.

6.3.1.4 Die Gruppe eine Verbands-Änderung vornimmt.

6.3.1.4.1 In diesem Fall muss die Gruppe ordentlich neu aufgenommen werden.

6.3.2 Der Ausschluss einer Gruppe kann jederzeit beantragt werden.

7. Stadtjugendring Ulm

7.1 Vollversammlung

7.1.1 Vor jeder Stadtjugendring-Vollversammlung trifft sich die
Delegiertenversammlung.

7.1.2 Hierbei werden mögliche Interessenkonflikte besprochen.

7.1.2.1 Bei Unstimmigkeiten wird ordentlich abgestimmt.

7.1.3 Auf Antrag ist die Stimmverteilung im Voraus demokratisch festzulegen.

7.1.4 Abstimmende haben bei der Stadtjugendring Vollversammlung primär im Interesse der Pfadfinder Rings abzustimmen.

7.2 Leitbild und Satzung

7.2.1 Der Ulmer Pfadfinder-Ring bekennt sich zum Leitbild (Stand 2023) des Stadtjugendring.

7.2.2 Der Ulmer Pfadfinder-Ring bekennt sich zu der Satzung (Stand 2023) des Stadtjugendring.

8. Absichtserklärung

8.1 Der Pfadfinder Ring verfolgt das Ziel, zu jeder Zeit mindestens ein Mitglied in dem Hauptausschuss des Stadtjugendring zu stellen.

8.2 Der Pfadfinder Ring hat das Ziel jede die ihm zur Verfügung stehende Stimme wahrzunehmen.

8.3 Der Ring möchte für Interessen einzelner Mitgliedsgruppen mit allen ihm zur Verfügung stehenden Mitteln einstehen.

\newpage

Unterschriften
\\

Saint-Exupéry (DPSG)
\\
(Ort, Datum, Name, Unterschrift) (Ort, Datum, Name, Unterschrift)
\\

Stamm-Nord (DPSG)
\\
(Ort, Datum, Name, Unterschrift) (Ort, Datum, Name, Unterschrift)
\\

Thor Heyerdahl (BdP)
\\
(Ort, Datum, Name, Unterschrift) (Ort, Datum, Name, Unterschrift)
\\

Ulm-Söflingen (DPSG)
\\
(Ort, Datum, Name, Unterschrift) (Ort, Datum, Name, Unterschrift)
\\

Weiße Rotte (BdP)
\\
(Ort, Datum, Name, Unterschrift) (Ort, Datum, Name, Unterschrift)
