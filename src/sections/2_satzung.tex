\begin{Large}
    \textbf{ORDNUNG}
\end{Large}
\\

% How to "Gendern" ?
% Sind wir der Pfadfinderring Ulm, oder der Pfadfinder*Innen-Ring Ulm?

% FRAGE an Verena S.
% TODO Wo sind "Nummerierungsfehler" im Dokument?
%      -> Wenn es nur einen Unterpunkt gibt, dann einfach direkt dazuschreiben.

\begin{legal}
    \item Name und Geschäftsjahr der Interessengemeinschaft
        \begin{legal}
            \item Der Name der Interessengemeinschaft lautet "Ulmer Pfadfinderring".
            \item Das Geschäftsjahr der Interessengemeinschaft entspricht dem Kalenderjahr.
        \end{legal}
    \item Zweck der Interessengemeinschaft
        \begin{legal}
            \item Die Interessengemeinschaft ist ein Zusammenschluss mehrerer in Ulm angesiedelter
                  Pfadfinder-Gruppierungen mit dem Ziel der Erziehung junger Menschen nach den 
                  Grundsätzen der Internationalen Pfadfinderbewegung. Dies geschieht im Zusammenhang 
                  mit allen Trägern der Jugenderziehung zu freien, verantwortungsbewussten, und 
                  toleranten Bürgerinnen und Bürgern eines demokratischen Staates.
                  % TODO Was ist mit "im Zusammenhang mit allen Trägern" gemeint?
            \item Die Interessengemeinschaft ist parteipolitisch und konfessionell neutral. 
                  Sie bekennt sich zum demokratischen Rechtsstaat und zum Grundgesetz der 
                  Bundesrepublik Deutschland.
            \item Ziel der Interessengemeinschaft ist das Planen, Durchführen und Abschließen von 
                  Veranstaltungen für lokale, nationale, und internationale Treffen von 
                  Kindern und Jugendlichen.
            % TODO Muss hier rein, dass wir kein Vermoegen verwalten?
            %      -> Wir sind eine reine Interessengemeinschaft / ein Dachverband,
            %         wir verwalten kein Geld. Muss das hier hinein?
        \end{legal}
    \item Organe der Interessengemeinschaft
        \begin{legal}
            \item Die Organe der Interessengemeinschaft sind...
                \begin{legal}
                    \item der Sprecher / die Sprecherin.
                    \item die Delegiertenversammlung.
                \end{legal}
        \end{legal}
    \item Sprecher oder Sprecherin
        \begin{legal}
            \item Der Sprecher / die Sprecherin wird durch die Delegiertenversammlung 
                  demokratisch gewählt.
                  % TODO Auf welche Zeit? 
                  %      - Lebenslang? 
                  %      - Bis zur Abwahl?
                  %      - fuer einen Zeitraum? (e.g. 1 Jahr)
            \item Die Wiederwahl des Sprechers / der Sprecherin ist zulässig.
            \item Bei Ausscheiden des Sprechers / der Sprecherin wählt die Delegiertenversammlung 
                  eine Nachfolge.
            \item Die Abwahl des Sprechers / der Sprecherin ist auf Antrag durch die 
                  Delegiertenversammlung möglich.
                \begin{legal}
                    \item Der Antrag gilt bei mindestens 75\% Zustimmung als angenommen.
                \end{legal}
            \item Der Sprecher / die Sprecherin ist an die Beschlüsse der Delegiertenversammlung 
                  unter Beachtung der Satzung gebunden.
            \item Der Sprecher / die Sprecherin hat folgende Aufgaben:
                  \begin{legal}
                        \item Die Weitergabe von Terminen und Aufgaben an die Gruppen.
                        \item Die Terminfindung für Delegiertenversammlungen.
                        %     ^ TODO Mit welcher Vorlaufzeit muss eingeladen werden?
                        %            Wer muss alles da sein, um stimmberechtigt zu sein?
                        %            (wie viele Staemme, & welche Personen aus diesen Staemmen?)
                  \end{legal}
            \item Der Sprecher / die Sprecherin fungiert als vermittlelnde Person zwischen dem 
                  Ulmer Pfadfinderring und dem Stadtjugendring Ulm e.V..
            \item Der Sprecher / die Sprecherin fungiert als Ansprechpartner und vermittlelnde 
                  Person gegenüber Dritten.
        \end{legal}
    \item Delegiertenversammlung
        \begin{legal}
            \item Die Delegiertenversammlung ist das oberste beschlussfassende Organ
                  des Ulmer Pfadfinderrings.
            \item Die Delegiertenversammlung entscheidet grundsätzlich mit einfacher Mehrheit der 
                  in der Versammlung abgegebenen Stimmen (Stimmenmehrheit).
                  % TODO Ab wann ist man beschlussfaehig?
                  %      - Kann ich eine Versammlung einberufen, & alleine bestimmen?
                  %      - Muss von jeder Gruppierung jemand anwesend sein? 
                  %        Und: Wer muss anwesend sein? StaVos? Von ihnen bestimmte Vertreter?
                  %      - In welchem Vorlauf muss zu einer Versammlung eingeladen werden?
                  %        (Kann ich jetzt einladen, & nachher abhalten? -> Ohne Euch?)
            \item Im Falle einer Stimmengleichheit gilt der Antrag als abgelehnt.
            \item Die Delegiertenversammlung tritt mindestens einmal jährlich zusammen.
            \item Die Delegiertenversammlung setzt sich aus jeweils zwei Vertretern oder 
                  Vertreterinnen aus den Mitgliedsgruppen zusammen.
                \begin{legal}
                    \item Die Delegierten können von jeder Gruppe frei bestimmt werden.
                          % TODO Von wem?
                    \item Geschieht das nicht, sind die gewählten Vorstände der Gruppen kraft Amtes, 
                          hierarchisch absteigend, delegiert.
                \end{legal}
            \item Jedes Mitglied der Delegiertenversammlung muss das 16. Lebensjahr vollendet haben.
            \item Anträge
                \begin{legal}
                    \item Jede Gruppe kann jederzeit in schriftlicher oder mündlicher Form Anträge 
                          anbringen.
                    \item Einfache Anträge werden in der folgenden Delegiertenversammlung 
                          bearbeitet.
                \end{legal}
            \item Bei einem Eilantrag kann eine außerordentliche Delegiertenversammlung 
                  einberufen werden.
                  \begin{legal}
                      \item Ein Antrag ist eilig, wenn dieser einen unaufschiebbaren und den 
                            Pfadfinderring nicht unerheblich betreffenden Inhalt vorweist.
                            % TODO Wie ist das definiert?
                  \end{legal}
        \end{legal}
    \item Eintritt, Austritt und Ausschluss aus dem Ulmer Pfadfinderring
        \begin{legal}
            \item Eintritt
                \begin{legal}
                    \item Ein Aufnahmeantrag kann nur schriftlich erfolgen.
                        \begin{legal}
                            \item Der Antrag kann formlos gestellt werden.
                        \end{legal}
                    \item Über Aufnahmeanträge entscheidet die Delegiertenversammlung.
                        \begin{legal}
                            \item Der Aufnahme-Antrag gilt bei mindestens 75\% Zustimmung als 
                                  angenommen.
                        \end{legal}
                    \item Die antragstellende Gruppe muss sich vor Beitritt zu der Satzung des 
                          Ulmer Pfadfinderrings bekennen.
                \end{legal}
            \item Austritt
                \begin{legal}
                    \item Eine Austrittserklärung kann nur schriftlich erfolgen.
                    \item Bei Zustimmung durch die Delegiertenversammlung (Eilantrag) kann der 
                          Austritt sofort umgesetzt werden.
                    \item Andernfalls gilt die nächste ordentliche Delegiertenversammlung als 
                          Austrittstermin.
                \end{legal}
            \item Ausschluss
                \begin{legal}
                    \item Ein Ausschluss kann erfolgen, wenn...
                        \begin{legal}
                            \item die Gruppe nicht mehr existiert.
                            \item die Gruppe seit mindestens einem Jahr keine Angebote für Kinder 
                                  und/oder Jugendliche anbietet.
                            \item die Arbeit der Gruppe dem Leitbild oder der Satzung des Ulmer 
                                  Pfadfinderrings widerspricht.
                            \item die Gruppe eine Verbandsänderung vornimmt.
                                \begin{legal}
                                    \item In diesem Fall muss die Gruppe ordentlich neu aufgenommen 
                                          werden.
                                          % ^ TODO Fix list item indentation here.
                                \end{legal}
                        \end{legal}
                        % TODO Welche Mehrheit braucht man beim Ausschluss?
                    \item Der Ausschluss einer Gruppe kann jederzeit beantragt werden.
                \end{legal}
        \end{legal}
    \item Stadtjugendring Ulm e.V.
        % TODO Warum ist "SJR" eine "falsche Abkuerzung" ?
        \begin{legal}
            \item Vollversammlung
                \begin{legal}
                    \item Vor jeder Vollversammlung des Stadtjugendring Ulm e.V. trifft sich die 
                          Delegiertenversammlung.
                    \item Hierbei werden mögliche Interessenkonflikte besprochen.
                        \begin{legal}
                            \item Bei Unstimmigkeiten wird ordentlich abgestimmt.
                        \end{legal}
                    \item Auf Antrag ist die Stimmverteilung im Voraus demokratisch festzulegen.
                          % TODO Was ist hiermit gemeint?
                    \item Der Sprecher / die Sprecherin hat in der Vollversammlung des 
                          Stadtjugendring Ulm e.V. primär im Interesse des Ulmer Pfadfinderrings 
                          abzustimmen.
                \end{legal}
            \item Leitbild und Satzung
                \begin{legal}
                    \item Der Ulmer Pfadfinderring bekennt sich zur aktuellsten Fassung 
                          des Leitbild des Stadtjugendring Ulm e.V.
                    \item Der Ulmer Pfadfinderring bekennt sich zur aktuellsten Fassung der 
                          Satzung des Stadtjugendring Ulm e.V.
                \end{legal}
        \end{legal}
    \item Absichtserklärung
        \begin{legal}
            \item Der Ulmer Pfadfinderring verfolgt das Ziel, zu jeder Zeit über mindestens einen 
                  Vertreter / eine Vertreterin im Hauptausschuss des Stadtjugendring Ulm e.V. zu 
                  verfügen.
            \item Der Ulmer Pfadfinderring hat das Ziel, jede ihm zur Verfügung stehende Stimme 
                  wahrzunehmen.
                  % TODO Was ist hiermit gemeint?
            \item Der Ulmer Pfadfinderring wird sich mit allen ihm zur Verfügung stehenden 
                  Mitteln für Interessen einzelner Mitgliedsgruppen einsetzen.
        \end{legal}
\end{legal}
