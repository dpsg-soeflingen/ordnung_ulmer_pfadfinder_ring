\begin{Large}
    \textbf{SATZUNG}
\end{Large}
\\

\begin{legal}
    \item Name und Geschäftsjahr der Interessengemeinschaft
        \begin{legal}
            \item Der Name der Interessengemeinschaft ist "Ulmer Pfadfinder-Ring".
            \item Das Geschäftsjahr der Interessengemeinschaft entspricht dem Kalenderjahr.
        \end{legal}
    \item Zweck der Interessengemeinschaft
        \begin{legal}
            \item Die Interessengemeinschaft ist ein Zusammenschluss von Pfadfinder-Gruppierungen 
                  zur Erziehung junger Menschen nach den Grundsätzen der Internationalen 
                  Pfadfinderbewegung im Zusammenhang mit allen Trägern der Jugenderziehung zu 
                  freien, verantwortungsbewussten und toleranten Bürgerinnen und Bürgern eines 
                  demokratischen Staates.
            \item Die Interessengemeinschaft ist parteipolitisch und konfessionell neutral. 
                  Sie bekennt sich zum demokratischen Rechtsstaat und zum Grundgesetz der 
                  Bundesrepublik Deutschland.
            \item Ziel der Interessengemeinschaft ist das Planen, Durchführen und Abschließen von 
                  Veranstaltungen zum regionalen, nationalen, und internationalen Treffen von 
                  Kindern und Jugendlichen.
            % TODO Muss hier rein, dass wir kein Vermoegen verwalten?
            %      -> Wir sind eine reine Interessengemeinschaft / ein Dachverband,
            %         wir verwalten kein Geld. Muss das hier hinein?
        \end{legal}
    \item Organe der Interessengemeinschaft
        \begin{legal}
            \item Die Organe der Interessengemeinschaft sind
                \begin{legal}
                    \item Der Sprecher oder die Sprecherin.
                    \item Die Delegiertenversammlung.
                \end{legal}
        \end{legal}
    \item Sprecher oder Sprecherin
        \begin{legal}
            \item Der Sprecher oder die Sprecherin wird durch die Delegiertenversammlung 
                  demokratisch gewählt.
                  % TODO Auf welche Zeit? 
                  %      - Lebenslang? 
                  %      - Bis zur Abwahl?
                  %      - fuer einen Zeitraum? (e.g. 1 Jahr)
            \item Die Wiederwahl des Sprechers oder der Sprecherin ist zulässig.
            \item Bei Ausscheiden des Sprechers oder der Sprecherin wählt die Delegiertenversammlung 
                  eine Nachfolge.
            \item Die Abwahl des Sprechers oder der Sprecherin ist auf Antrag durch die 
                  Delegiertenversammlung möglich.
                \begin{legal}
                    \item Der Antrag gilt bei mindestens 75\% Zustimmung als Angenommen.
                \end{legal}
            \item Der Sprecher oder die Sprecherin ist an die Beschlüsse der Delegiertenversammlung 
                  unter Beachtung der Satzung gebunden.
            \item Der Sprecher oder die Sprecherin hat folgende Aufgaben:
                  \begin{legal}
                        \item Weitergabe von Terminen und Aufgaben an die Gruppen.
                        \item Terminfindung für Delegiertenversammlungen.
                  \end{legal}
            \item Der Sprecher oder die Sprecherin fungiert als Vermittler/-in zwischen dem Ulmer 
                  Pfadfinder-Ring und dem Stadtjugendring Ulm e.V. (SJR).
            \item Der Sprecher oder die Sprecherin ist Ansprechpartner/-in und Vermittler/-in 
                  gegenüber Dritten.
        \end{legal}
    \item Delegiertenversammlung
        \begin{legal}
            \item Die Delegiertenversammlung ist das oberste beschlussfassende Organ.
            \item Die Delegiertenversammlung entscheidet grundsätzlich mit einfacher Mehrheit der 
                  in der Versammlung abgegebenen Stimmen(Stimmenmehrheit).
                  % TODO Ab wann ist man beschlussfaehig?
                  %      - Kann ich eine Versammlung einberufen, & alleine bestimmen?
                  %      - Muss von jeder Gruppierung jemand anwesend sein? 
                  %        Und: Wer muss anwesend sein? StaVos? Von ihnen bestimmte Vertreter?
                  %      - In welchem Vorlauf muss zu einer Versammlung eingeladen werden?
                  %        (Kann ich jetzt einladen, & nachher abhalten? -> Ohne Euch?)
            \item Im Falle einer Stimmengleichheit gilt der Antrag als abgelehnt.
            \item Die Delegiertenversammlung tritt mindestens einmal jährlich zusammen.
            \item Die Delegiertenversammlung setzt sich aus jeweils zwei Vertretern oder 
                  Vertreterinnen aus den Mitgliedsgruppen zusammen.
                \begin{legal}
                    \item Die Delegierten können von jeder Gruppe frei bestimmt werden.
                    \item Geschieht das nicht, sind die gewählten Vorstände der Gruppen Kraft Amtes, 
                          hierarchisch absteigend, delegiert.
                \end{legal}
            \item Ein Mitglied der Delegiertenversammlung muss das 16. Lebensjahr vollendet haben.
            \item Anträge
                \begin{legal}
                    \item Jede Gruppe kann jederzeit in schriftlicher oder mündlicher Form Anträge 
                          anbringen.
                    \item Einfache Anträge werden in der folgenden Delegiertenversammlung 
                          bearbeitet.
                    \item Bei einem Eilantrag kann eine außerordentliche Delegiertenversammlung 
                          einberufen werden.
                        \begin{legal}
                            \item Ein Antrag ist eilig, wenn dieser einen unaufschiebbaren und den 
                                  Pfadfinder-Ring nicht unerheblich betreffenden Inhalt vorweist.
                        \end{legal}
                \end{legal}
        \end{legal}
    \item Eintritt, Austritt und Ausschluss aus dem Ulmer Pfadfinder-Ring
        \begin{legal}
            \item Eintritt
                \begin{legal}
                    \item Ein Aufnahmeantrag kann nur schriftlich erfolgen.
                        \begin{legal}
                            \item Der Antrag kann formlos gestellt werden.
                        \end{legal}
                    \item Über Aufnahmeanträge entscheidet die Delegiertenversammlung.
                        \begin{legal}
                            \item Der Aufnahme-Antrag gilt bei mindestens 75\% Zustimmung als 
                                  Angenommen.
                        \end{legal}
                    \item Die antragstellende Gruppe muss sich vor Beitritt zu der Satzung des 
                          "Ulmer Pfadfinder-Ring" bekennen.
                \end{legal}
            \item Austritt
                \begin{legal}
                    \item Eine Austrittserklärung kann nur schriftlich erfolgen.
                    \item Bei Zustimmung durch die Delegiertenversammlung (Eilantrag) kann der 
                          Austritt sofort umgesetzt werden.
                    \item Andernfalls gilt die nächste ordentliche Delegiertenversammlung als 
                          Austrittstermin.
                \end{legal}
            \item Ausschluss
                \begin{legal}
                    \item Ein Ausschluss kann erfolgen, wenn:
                        \begin{legal}
                            \item Die Gruppe nicht mehr existiert
                            \item Die Gruppe seit mindestens einem Jahr keine Angebote für Kinder 
                                  und Jugendliche anbietet.
                            \item Die Arbeit der Gruppe dem Leitbild oder der Satzung des Ulmer 
                                  Pfadfinder-Rings widerspricht.
                            \item Die Gruppe eine Verbands-Änderung vornimmt.
                                \begin{legal}
                                    \item In diesem Fall muss die Gruppe ordentlich neu aufgenommen 
                                          werden.
                                          % ^ TODO Fix list item indentation here.
                                \end{legal}
                        \end{legal}
                    \item Der Ausschluss einer Gruppe kann jederzeit beantragt werden.
                \end{legal}
        \end{legal}
    \item Stadtjugendring Ulm e.V.
        \begin{legal}
            \item Vollversammlung
                \begin{legal}
                    \item Vor jeder Vollversammlung des Stadtjugendring Ulm e.V. trifft sich die 
                          Delegiertenversammlung.
                    \item Hierbei werden mögliche Interessenkonflikte besprochen.
                        \begin{legal}
                            \item Bei Unstimmigkeiten wird ordentlich abgestimmt.
                        \end{legal}
                    \item Auf Antrag ist die Stimmverteilung im Voraus demokratisch festzulegen.
                    \item Abstimmende haben bei der Vollversammlung des Stadtjugendring e.V. primär 
                          im Interesse der Pfadfinder-Rings abzustimmen.
                \end{legal}
            \item Leitbild und Satzung
                \begin{legal}
                    \item Der Ulmer Pfadfinder-Ring bekennt sich zum Leitbild (Stand 2023) des 
                          Stadtjugendring Ulm e.V.
                    \item Der Ulmer Pfadfinder-Ring bekennt sich zu der Satzung (Stand 2023) des 
                          Stadtjugendring Ulm e.V.
                \end{legal}
        \end{legal}
    \item Absichtserklärung
        \begin{legal}
            \item Der Ulmer Pfadfinder-Ring verfolgt das Ziel, zu jeder Zeit mindestens einen 
                  Vertreter / eine Vertreterin in dem Hauptausschuss des Stadtjugendring Ulm e.V. 
                  zu stellen.
            \item Der Ulmer Pfadfinder-Ring hat das Ziel jede ihm zur Verfügung stehende Stimme 
                  wahrzunehmen.
            \item Der Ulmer Pfadfinder-Ring möchte sich mit allen ihm zur Verfügung stehenden 
                  Mitteln für Interessen einzelner Mitgliedsgruppen einsetzen.
        \end{legal}
\end{legal}
