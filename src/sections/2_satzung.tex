\begin{Large}
    \textbf{ORDNUNG}
\end{Large}
\\

% How to "Gendern" ?
% Sind wir der Pfadfinderring Ulm, oder der Pfadfinder*Innen-Ring Ulm?

% FRAGE an Verena S.
% TODO Wo sind "Nummerierungsfehler" im Dokument?
%      -> Wenn es nur einen Unterpunkt gibt, dann einfach direkt dazuschreiben.

\begin{legal}
    \item Name und Geschäftsjahr der Interessengemeinschaft
        \begin{legal}
            \item Der Name der Interessengemeinschaft lautet "Ulmer Pfadfinderring".
            \item Das Geschäftsjahr der Interessengemeinschaft entspricht dem Kalenderjahr.
        \end{legal}
    \item Zweck der Interessengemeinschaft
        \begin{legal}
            \item Die Interessengemeinschaft ist ein Zusammenschluss mehrerer in Ulm angesiedelter
                  Pfadfinder-Gruppierungen, die das Ziel der Erziehung junger Menschen nach den 
                  Grundsätzen der Internationalen Pfadfinderbewegung haben. Dies geschieht im Zusammenhang 
                  mit allen Trägern der Jugendarbeit zu freien, verantwortungsbewussten und 
                  toleranten Mitgliedern der Gesellschaft eines demokratischen Staates.
            \item Die Interessengemeinschaft ist parteipolitisch und konfessionell neutral. 
                  Sie bekennt sich zum demokratischen Rechtsstaat und zum Grundgesetz der 
                  Bundesrepublik Deutschland.
            \item Ein Ziel der Interessengemeinschaft ist das Planen, Durchführen und Abschließen 
                  von Veranstaltungen für lokale, nationale, und internationale Treffen von 
                  Kindern und Jugendlichen.
            \item Ausdruecklich Kein Zweck der Interessengemeinschaft ist die Verwaltung von        
                  Vermoegen.
                  % TODO Umlaut oe und ue
        \end{legal}
    \item Organe der Interessengemeinschaft
        \begin{legal}
            \item Die Organe der Interessengemeinschaft sind:
                \begin{itemize}
                    \item die vertretende Person und
                    \item die Delegiertenversammlung.
                \end{itemize}
        \end{legal}
    \item Die vertretende Person
        \begin{legal}
            \item Die vertretende Person wird durch die Delegiertenversammlung 
                  demokratisch gewählt.
            \item Die vertretende Person wird auf einen Zeitraum von 2 Jahren gewählt.
            \item Die Wiederwahl der vertretenden Person ist zulässig.
            \item Bei Ausscheiden der vertretenden Person wählt die Delegiertenversammlung 
                  eine Nachfolge.
            \item Die Abwahl der vertretenden Person ist auf Antrag durch die 
                  Delegiertenversammlung möglich. Der Antrag gilt bei mindestens 75\% Zustimmung der
                  Delegiertenversammlung als angenommen.
            \item Die vertretende Person ist an die Beschlüsse der Delegiertenversammlung 
                  unter Beachtung der Ordnung gebunden.
            \item Die vertretende Person hat folgende Aufgaben:
                  \begin{itemize}
                        \item die Weitergabe von Terminen und Aufgaben an die Gruppen und
                        \item die Terminfindung für Delegiertenversammlungen.
                  \end{itemize}
            \item Die vertretende Person fungiert als vermittlelnde Person zwischen dem 
                  Ulmer Pfadfinderring und dem Stadtjugendring Ulm e.V.
            \item Die vertretende Person fungiert als Ansprechperson und vermittlelnde 
                  Person gegenüber Dritten.
        \end{legal}
    \newpage
    \item Die Delegiertenversammlung
        \begin{legal}
            \item Die Delegiertenversammlung ist das oberste beschlussfassende Organ
                  des Ulmer Pfadfinderrings.
            \item Die Delegiertenversammlung setzt sich aus jeweils maximal zwei 
                  vertretenden Personen aus den Mitgliedsgruppen zusammen.
                \begin{legal}
                    \item Die Delegierten können von jeder Gruppe frei bestimmt werden 
                    \item Geschieht das nicht, sind die gewählten Vorstände der Gruppen kraft Amtes, 
                          hierarchisch absteigend, delegiert.
                    \item Die Delegierten müssen der vertretenden Person des Ulmer 
                          Pfadfinderrings vor der Delegiertenversammlung mitgeteilt werden.
                \end{legal}
            \item Die Delegiertenversammlung entscheidet grundsätzlich mit einfacher Mehrheit der 
                  abgegebenen Stimmen (Stimmenmehrheit).
            \item Im Falle einer Stimmengleichheit gilt der Antrag als abgelehnt.
            \item Die Delegiertenversammlung tritt mindestens einmal jährlich zusammen.
                  Eingeladen werden muss mindestens zwei Wochen vorher.
            \item Jedes Mitglied der Delegiertenversammlung muss das 16. Lebensjahr vollendet haben.
            \item Anträge
                \begin{legal}
                    \item Jede Gruppe kann jederzeit in schriftlicher oder mündlicher Form Anträge 
                          anbringen.
                    \item Einfache Anträge werden in der folgenden Delegiertenversammlung 
                          bearbeitet.
                \end{legal}
            \item Bei einem Eilantrag kann eine außerordentliche Delegiertenversammlung 
                  einberufen werden.
                  \begin{legal}
                      \item Ein Antrag ist eilig, wenn dieser einen unaufschiebbaren und den 
                            Pfadfinderring nicht unerheblich betreffenden Inhalt vorweist.
                  \end{legal}
        \end{legal}
    \item Eintritt, Austritt und Ausschluss aus dem Ulmer Pfadfinderring
        \begin{legal}
            \item Eintritt
                \begin{legal}
                    \item Ein Aufnahmeantrag kann nur schriftlich erfolgen.
                          Der Antrag kann formlos gestellt werden.
                    \item Über Aufnahmeanträge entscheidet die Delegiertenversammlung.
                          Der Aufnahme-Antrag gilt bei mindestens 75\% Zustimmung 
                          der Delegiertenversammlung als angenommen.
                    \item Die antragstellende Gruppe muss sich vor Beitritt zu der Ordnung des 
                          Ulmer Pfadfinderrings bekennen.
                \end{legal}
            \item Austritt
                \begin{legal}
                    \item Eine Austrittserklärung kann nur schriftlich erfolgen.
                    \item Bei Zustimmung durch die außerordentliche Delegiertenversammlung 
                          (Eilantrag) kann der Austritt sofort umgesetzt werden.
                    \item Andernfalls gilt die nächste ordentliche Delegiertenversammlung als 
                          Austrittstermin.
                \end{legal}
            \item Ausschluss
                \begin{legal}
                    \item Ein Antrag auf Ausschluss kann erfolgen, wenn ...
                        \begin{legal}
                            \item die Gruppe nicht mehr existiert.
                            \item die Gruppe seit mindestens einem Jahr keine Angebote für Kinder 
                                  oder Jugendliche anbietet.
                            \item die Arbeit der Gruppe dem Leitbild oder der Ordnung des Ulmer 
                                  Pfadfinderrings widerspricht.
                            \item die Gruppe eine Verbandsänderung vornimmt. In diesem Fall muss 
                                  die Gruppe ordentlich neu aufgenommen werden.
                        \end{legal}
                        % TODO Welche Mehrheit braucht man beim Ausschluss? -> 50%
                    \item Der Ausschluss einer Gruppe kann jederzeit beantragt werden.
                \end{legal}
        \end{legal}
    \item Stadtjugendring Ulm e.V.
        \begin{legal}
            \item Vollversammlung
                \begin{legal}
                    \item Vor jeder Vollversammlung des Stadtjugendrings Ulm e.V. trifft sich die 
                          Delegiertenversammlung fristgerecht.
                    \item Hierbei werden mögliche Interessenkonflikte besprochen.
                          Bei Unstimmigkeiten wird ordentlich abgestimmt.
                    \item Auf Wunsch ist die Stimmverteilung im Voraus demokratisch festzulegen.
                    \item Die vier Delegierten haben in der Vollversammlung des 
                          Stadtjugendrings Ulm e.V. primär im Interesse des Ulmer Pfadfinderrings 
                          abzustimmen.
                \end{legal}
            \item Leitbild und Satzung
                \begin{legal}
                    \item Der Ulmer Pfadfinderring bekennt sich zur aktuellsten Fassung 
                          des Leitbild des Stadtjugendrings Ulm e.V.
                    \item Der Ulmer Pfadfinderring bekennt sich zur aktuellsten Fassung der 
                          Satzung des Stadtjugendrings Ulm e.V.
                \end{legal}
        \end{legal}
    \item Absichtserklärung
        \begin{legal}
            \item Der Ulmer Pfadfinderring verfolgt das Ziel, zu jeder Zeit mindestens eine 
                  vertetende Person im Hauptausschuss des Stadtjugendrings Ulm e.V. 
                  und in anderen Gremien zu stellen.
            \item Der Ulmer Pfadfinderring hat das Ziel, jede ihm zur Verfügung stehende Stimme 
                  wahrzunehmen.
            \item Der Ulmer Pfadfinderring wird sich mit allen ihm zur Verfügung stehenden 
                  Mitteln für Interessen einzelner Mitgliedsgruppen einsetzen.
        \end{legal}
\end{legal}
