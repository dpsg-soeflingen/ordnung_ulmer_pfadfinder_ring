LEITBILD
\\

Das große Ziel der pfadfinderischen Arbeit ist es, Kinder und Jugendliche zu stärken und sie zu 
befähigen, ihre Potenziale so auszuschöpfen, dass sie als verantwortungsbewusste Bürgerinnen und 
Bürger die Welt mitgestalten können. Folgende Ziele verfolgt der Ulmer Pfadfinder-Ring:
\\

"look at the boy" – "look at the girl" – Das Zitat "look at the child" von Lord Robert Baden-Powell, 
dem Gründer der Weltpfadfinderbewegung, fasst die Grundlage für die Arbeit mit Kindern und 
Jugendlichen nach dem pfadfinderischen Verständnis zusammen: Der Entwicklungsstand und die 
Lebenswirklichkeit der Kinder und Jugendlichen werden innerhalb der Gruppenarbeit stets 
berücksichtigt. Im Grundsatz "look at the girl" betrachten die Mitglieder im Ring Deutscher 
Pfadfinderinnenverbände speziell die Mädchen und jungen Frauen.
\\

"Learning by doing" – In unseren Gruppen entwickeln Kinder und Jugendliche ihre Fähigkeiten durch 
eigenes Ausprobieren, durch Mitbestimmung und eigene Entscheidungen. Sie lernen durch die Übernahme 
von Aufgaben Verantwortung zu tragen, sie lernen partnerschaftlich zu handeln und sich über 
gemeinsame Erfolge zu freuen. Gleichzeitig erlernen sie soziales Verhalten durch das Akzeptieren 
von (Spiel-)Regeln.
\\

Demokratie lernen im Handeln – In vielen Projekten befähigt die pfadfinderische Methode Kinder und 
Jugendliche, soziale und politische Zusammenhänge zu erkennen, sich zu orientieren und ihre 
Interessen solidarisch mit anderen zu vertreten – auf lokaler, nationaler und internationaler Ebene. 
In den unterschiedlichen Gremien der Ortsgruppen der Verbände – der sogenannten Stämme – und ihrer 
Dachorganisationen wird Interessenvertretung und politisches Handeln nicht nur von Leiterinnen und 
Leitern, sondern schon von den Jüngsten eingeübt.
\\

Gerechtigkeit leben – Pfadfinder*innen treten für eine gerechte Welt ein. Dabei fangen sie zuerst 
bei sich an, vergessen dabei aber nicht die Menschen in anderen Ländern. Mädchen und Jungen sind bei 
den Pfadfinder*innen selbstverständlich gleichberechtigt und treten darüber hinaus dafür ein, dass 
sie in der Gesellschaft gleiche Chancen haben.
\\

Natur und Umwelt – Ein Höhepunkt im pfadfinderischen Jahreslauf findet draußen unter freiem Himmel 
statt: In Zeltlagern und "auf Fahrt" leben sie in und mit der Natur. Hier erleben Kinder und 
Jugendliche, mit wie wenig sie auskommen können und was wirklich von Bedeutung ist. So sind 
natürlich auch Nachhaltigkeit und Umweltschutz zwei große Themen, die bereits den Kleinen im 
Gruppenalltag beispielsweise bei einer Müllsammelaktion durch aktives Handeln nahegebracht werden.
\\

Pfadfinden für alle – Der Ulmer Pfadfinder-Ring steht für gegenseitige Akzeptanz, für die Erziehung 
zum Frieden und den weltweiten Abbau von Ungerechtigkeit und Armut. Darum sind bei den Gruppen des 
Ulmer Pfadfinder-Ring Kinder und Jugendliche jeder nationalen, religiösen, ethnischen oder sozialen 
Zugehörigkeit herzlich willkommen.
